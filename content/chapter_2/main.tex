\chapter{Background}

\section{Why Zensor exist?}
Today, most often, technical data sheets coupled with the knowledge of a number of unique experienced individuals are used to determine when maintenance is required for the asset. 
Product quality only becomes an issue when customers start complaining and repairs are done when it's already far too late.
All of these puts tremendous strain on the people responsible for asset, while it could be avoided.
Unexpected shutdowns are costly and very demanding for the workforce involved; a possible solution would be making assets smart in order to increase the availability.
The only way to validate the actual health is by having a continuous look at a broad data set and adding a specific multi-aspect monitoring setup 
consisting of different sensor types that follow the behavior of the assets general state-of-health.
\subsection{About the company}
Zensor \cite{Misc:zensor_official_website} provides full, integrated and intelligent monitoring solutions for the industrial production, renewable energy and infrastructure sectors.
This allows to Zensor's customers to convert the potential contained in the world of IoT and Industry 4.0 into their reality. 
The company enable digitalization, but with an interface oriented towards the real human: an expert solution without the need for internal experts.
It provides not only monitoring devices or data analysis, but offers a full, standardized and standalone end-to-end product that leverages the value contained in a subset of the following aspects:
\begin{itemize}
    \item operational efficiency
    \item predictive maintenance
    \item energy efficiency
    \item ageing and degradation
    \item safety 
\end{itemize}
A standard offering consists in several aspects (if required) ordered logically below; as such Zensor takes end-to-end responsibility in monitoring the health and efficiency of structures and processes.
\begin{enumerate}
    \item Hardware \{sensors and acquisition units\}
    \item Installation and Commissioning \{engineering and CAD\}
    \item Data Management \{data transfer, storage, coupling to existing data sources (SCADA, weather, operational...), data cleaning and treatment\}
    \item Analysis and Reporting \{predictions, trend and event detection, real-time reporting through online dashboards.\}
\end{enumerate}
\begin{table}[ht]
    \caption{Assets, Industries and Infrastructure for which Zensor has specific products}
    \begin{tabular}{@{}lll@{}}
    \toprule
    Asset                 & Industries                       & Infrastructure                       \\ \midrule
    Rolling cranes        & Metal Production                 & Offshore wind                        \\
    Grinders and crushers & Mining and Materials             & Rail                                 \\
    Flattener rollers     & Food Production                  & Civil Infrastructure                 \\
    Rolling mills         & Glass Production                 & Energy                               \\
    Conveyor belts        & Discrete Manufacturing                                                  \\
    Tunnels               & Textiles                                                                \\
    Chain transporters    &                                                                         \\
    Sieves                &                                                                         \\
    Bridges               &                                                                         \\ \bottomrule
\end{tabular}
\end{table}
\subsection{Philosophy}
\paragraph{Vision} 
Technological advance can only bring real value to society when the user-facing component is driven by Simplicity and Clarity for the end-user. 
Zensor sees this as the fastest and most certain route to a world where man-made structures affect the sustainability of our planet in the least possible way:
\begin{enumerate}
    \item they are intrinsically safe;
    \item their useful life is optimized to the maximum;
    \item their impact on environment and society is quantified and communicated.
\end{enumerate}

\paragraph{Mission}
Translate technological innovations in monitoring and analysis into easy-to-understand, tangible and relevant information 
that we share in the way tailored for either production managers, management or maintenance professionals…
As such we are the knowledgeable and easy-to-reach companion for owners and operators in making their assets increasingly safe, efficient and sustainable. If up to us, till eternity.


\section{Zensor Approach}
\section{Zensor Approach}\label{section:zensor_approach}
A Zensor's standard offering can consists, if required, in several aspects, ordered logically here:
\begin{enumerate}
	\item \texttt{Hardware}: sensors and acquisition units.
	\item \texttt{Installation and Commissioning}: engineering and CAD.
	\item \texttt{Data Management:} data transfer \& storage with coupling to existing data sources (SCADA, weather, operational...), data cleaning and treatment.
	\item \texttt{Analysis and Reporting}: predictions, trend and event detection, real-time reporting through online dashboards.
\end{enumerate}
As such Zensor takes end-to-end responsibility in monitoring the health and efficiency of structures and processes, and
we will develop these aspects in the following paragraphs.
% \todo{EB2DG: ``we'll dive'' troppo informale}
\begin{figure}[ht]
	\centering
	\begin{tikzcd}
		{Hardware} & Installation & {Management} & Analysis
		\arrow[tail, from=1-1, to=1-2]
		\arrow[tail, from=1-2, to=1-3]
		\arrow[tail, from=1-3, to=1-4]
	\end{tikzcd}
	\caption{Building blocks for full project}
	\label{tik:standar_offer}
\end{figure}

\paragraph{Hardware and/or other Existing Data sources}
Deriving valuable insights from a monitoring system states from the data:
\begin{itemize}
	\item[A] identifying and locating the relevant data in existing databases/data warehouse;
	\item[B] putting the right sensors, with appropriate settings, on the right positions and measurement conditions and, afterwards, reading them out in the optimal way;
\end{itemize}
Option A, B or both are available: it depends on whether enough data is available from the start.
All this is clearly defined in each asset-specific package.

\paragraph{Installation and Commissioning}
% \todo{EB2DG: non capisco la frase ``Push the button...'' DG:rimossa}
Initially links to the existing data sources are established and data gets ingested.
Where required a set of acquisition units and sensors is installed on the machine or structure.
After a final verification on the spot (SAT) the monitoring system is launched: the assets enter the \acl{IoT}.

\paragraph{Data Management}
Data is continuously streaming in from individual setups as well as historian sources.
Structuring, verifying and cleaning the data sets is an essential prerequisite to allow for a profound analysis afterwards: on your way to an automated, continuous and smart follow-up.

\paragraph{Analysis and Clear Reporting}
Advanced insights are unlocked using algorithms based on physics as well as big data approaches, as seen in Chapter~\ref{section:data_anal}
Clear dashboards, warnings and periodic reports inform the owner or line manager about the present state and upcoming issues. Surprises are avoided, standstills reduced.

\subsubsection{Flexibility}
Now that we are clear about the possible components of the \ac{SaaS}~\cite{Article:matthewlemerle_2012_its}, we can distinguish two possible different types of operational flow, as illustrated in Figures~\ref{tik:standar_offer} and~\ref{tik:ligth_offer}.
\begin{figure}[ht]
	\centering
	\begin{tikzcd}
		{Datasources} & Commissioning & {Management} & Analysis
		\arrow[tail, from=1-1, to=1-2]
		\arrow[tail, from=1-2, to=1-3]
		\arrow[tail, from=1-3, to=1-4]
	\end{tikzcd}
	\caption{Building blocks for light project}
	\label{tik:ligth_offer}
\end{figure}

The reasons for having two paths with separate initial phases are several. One important factor is the current global semiconductor and chip shortage~\cite{Article:bbcnews_2021_chip},
 which makes it difficult to procure \ac{IPCs} and sensors.
On the other hand the light package (Figure~\ref{tik:ligth_offer}) could represent a first step, to try the service with a limited effort.
Having said that, it has lower costs and timelines compared to the full package, but it complicates multi-source data management and the analysis that can be performed will be of lower granularity.
Table~\ref*{tab:phase_diff} provides a further comparison.
\begin{table}[ht]
	\begin{tabularx}{\textwidth}{>{\parskip1ex}X@{\kern4\tabcolsep}>{\parskip1ex}X}
		\toprule
		\hfil\bfseries Full
		 &
		\hfil\bfseries Light
		\\\cmidrule(r{3\tabcolsep}){1-1}\cmidrule(l{-\tabcolsep}){2-2}
		%% FULL, separated by empty line or \par
		The engineering team has to design and devise which sensors to use, where to place them and why.\par
		Hardware initial cost is high, and it requires weeks or months before it is collected and delivered to the Zennor offices.\par
		 &
		%% LIGHT, separated by empty line or \par
		Some sensors/devices are already in place and can be integrated into the service\par
		Possibility to manage client's independently metric collected data, with an additional effort in the third step\par
		\\\cmidrule(r{3\tabcolsep}){1-1}\cmidrule(l{-\tabcolsep}){2-2}
		%% FULL, separated by empty line or \par
		Installation phase requires several steps, including testing hardware and connection both at the factory (FAT) and on-site (SAT)\par
		 &
		%% LIGHT, separated by empty line or \par
		Necessary and important information, such as network protocol and connection information, to set up links to existing data sources are collected.\par
		\\\bottomrule
	\end{tabularx}
	\caption[1° and 2° Phases differences of Zensor's Approach]{1° \texttt{Hardware} vs \texttt{Datasources} and 2° \texttt{Installation} vs \texttt{Commissioning} Phases differences}
	\label{tab:phase_diff}
\end{table}

That said, to give you some sense of what the service might do, here is a non-exhaustive list of the main advanced monitoring features and metrics available:
\begin{enumerate}
	\item[$\blacksquare$]\texttt{Availability}: have a continuous idea of availability, automatically as the platform combines different input streams and contextual information.
	\item[$\blacksquare$]\texttt{Performance}: based on the data collected and machine-learning based methods for determining the operational condition the performance is calculated.
	\item[$\blacksquare$]\texttt{Warnings}: whenever values start to deviate, or data streams stop, warnings are sent. This avoids ``black holes'' in the insights of the production line or assets.
	\item[$\blacksquare$]\texttt{Quality}: coupling to existing databases or using human input fields the product quality is linked to operational process parameters.
	\item[$\blacksquare$]\texttt{Factory information systems}: such systems are crucial for obtaining operational excellence. When well managed they maximize efficiency and effectiveness. Automated data collection and advanced analysis makes this possible.
\end{enumerate}

\subsubsection[short]{Maintenance metrics}
A typical analysis involves performing a variety of calculations to obtain useful metrics, from static to maintenance; but what are the maintenance metrics anyway?
There are two categories of maintenance \ac{KPIs} which include the leading and lagging indicators. The leading indicators signal future events and the lagging indicators follow the past events.
We will give space to the latter category. For completeness here is how \acl{infra} platform could help you, asset owner, dealing with $MTTF$, $MTBF$, $OEE$.
\begin{itemize}
	\item[$\star$] $MTTF$: The \textit{Mean Time Till Failure} is tracked continuously, for each asset covered the overall ``disturbance free'' operation is displayed.
	\item[$\star$] $MTBF$: As events and operating conditions are automatically detected the \textit{Mean Time Between Failures} is determined continuously, giving a good insight on where optimization is possible.
	\item[$\star$] $OEE$: Using all parameters cited above the \textit{Overall Equipment Effectiveness} is determined, a major parameter for optimizing asset management strategies and future investments, with a huge cost savings potential.
\end{itemize}

\section{Stack/Pipeline in a nutshell}


% \section{Introduction to Analysis}
% To make dashboards happen, we are using Grafana. 
% One Grafana environment allows for multiple so-called "organizations". 
% Every project, thus, has one "organization" which is visible to the client (usually named after client) 
% and another one (usually named Zensor) that is used for internal purposes.

% \subsection{Loading speed of Grafana dashboards}

% \texttt{And how to keep it fast}

% Loading speed of a Grafana dashboard depends on 5 major things:

% \begin{enumerate}
%     \item  pre-selected and saved time window: the larger the time period you query, the longer it takes to open and display the contents'
%     \item  data frequency in the panels - in case of the very high frequency, non aggregated data, even if selected time period is minutes, it will take time to load
%     \item  number of panels with the data inside
%     \item  your database structure
%     \item  whether calculations have to happen inside the panel before the data is displayed
% \end{enumerate}

% \subsection{Operational statistics' dashboards}
% To make sure statistics page loads fast enough:

% \begin{itemize}
%     \item Take all calculations out of Grafana and only display measurement contents. Zensor Library module is available to calculate statistics on different data streams and different frequencies.
%     \item Along with the point above, make sure \textit{'time'} in the query is set to a dynamic interval and not grouped on, e.g. \textit{'1d'}
%     \item Use 'rows' to group panels together by topic and close the ones that don't have to be displayed immediately on dashboard load (and save it like that).
% \end{itemize}