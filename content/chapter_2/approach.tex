\subsection{What are the stages}
\[\begin{tikzcd}
	{Hardware} & Installation & {Data \space Management} & Analysis
	\arrow[tail, from=1-1, to=1-2]
	\arrow[tail, from=1-2, to=1-3]
	\arrow[tail, from=1-3, to=1-4]
\end{tikzcd}\]
\paragraph{Hardware and/or other Existing Data sources}
Sometimes not enough data is available from the beginning. Deriving valuable insights from a monitoring system states from the data:
identifying and locating the relevant data in existing databases/data warehouse or putting the right sensors,
with appropriate settings, on the right positions and measurement conditions;
afterwards reading them out in the optimal way. All of this are defined clearly in every asset-specific package.

\paragraph{Installation and Commissioning}
Push the button of Industry 4.0. Initially links to the existing data sources are established and data gets ingested. 
Where required a set of acquisition units and sensors is installed on the machine or structure.
After a final verification on the spot (SAT) the monitoring system is launched: the assets enter the IoT.

\paragraph{Data Management}
Data is continuously streaming in from individual setups as well as historian sources. 
Structuring, verifying and cleaning the data sets is an essential prerequisite to allow for a profound analysis afterwards: on your way to an automated, continuous and smart follow-up.

\paragraph{Analysis and Clear Reporting}
Advanced insights are unlocked using algorithms based on physics as well as big data approaches. 
Clear dashboards, warnings and periodic reports inform the owner or line manager about the present state and upcoming issues. Surprises are avoided, standstills reduced.

\subsection{Advanced features and metrics}
Here is a non-exhaustive list of the main monitoring metrics available 
\paragraph{Availability}
Have a continuous idea of availability, automatically as the platform combines different input streams and contextual information.
\paragraph{Performance}
Based on the data collected and machine-learning based methods for determining the operational condition the performance is calculated.
\paragraph{Warnings}
Whenever values start to deviate, or data streams stop, warnings are sent. This avoids 'black holes' in the insights of the production line or assets.
\paragraph{Quality}
Coupling to existing databases or using human input fields the product quality is linked to operational process parameters.
\paragraph{MTTF}
The \(Mean\space Time\space Till\space Failure\) is tracked continuously, for each asset covered the overall 'disturbance free' operation is displayed.
\paragraph{MTBF}
As events and operating conditions are automatically detected the \(Mean \space Time \space Between \space Failures\) is determined continuously, giving a good insight on where optimization is possible.
\paragraph{OEE}
Using all parameters cited above the \(Overall\space Equipment\space Effectiveness\) is determined, a major parameter for optimizing asset management strategies and future investments, with a huge cost savings potential.
\paragraph{Factory information systems}
Such systems are crucial for obtaining operational excellence. When well managed they maximize efficiency and effectiveness. Automated data collection and advanced analysis makes this possible.
