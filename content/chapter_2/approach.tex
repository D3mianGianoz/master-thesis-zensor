\section{Zensor Approach}
A Zensor's standard offering can consists, if required, in several aspects, ordered logically here:
\begin{enumerate}
	\item \texttt{Hardware}: sensors and acquisition units.
	\item \texttt{Installation and Commissioning}: engineering and CAD.
	\item \texttt{Data Management:} data transfer \& storage with coupling to existing data sources (SCADA, weather, operational...), data cleaning and treatment.
	\item \texttt{Analysis and Reporting}: predictions, trend and event detection, real-time reporting through online dashboards.
\end{enumerate}
as such Zensor takes end-to-end responsibility in monitoring the health and efficiency of structures and processes, we'll dive in these aspect in the following subsection.
\begin{figure}[ht]
	\begin{tikzcd}
		{Hardware} & Installation & {Management} & Analysis
		\arrow[tail, from=1-1, to=1-2]
		\arrow[tail, from=1-2, to=1-3]
		\arrow[tail, from=1-3, to=1-4]
	\end{tikzcd}
	\caption{Building blocks for full project}
	\label{tik:standar_offer}
\end{figure}

\paragraph{Hardware and/or other Existing Data sources}
Sometimes not enough data is available from the beginning. Deriving valuable insights from a monitoring system states from the data:
identifying and locating the relevant data in existing databases/data warehouse or putting the right sensors,
with appropriate settings, on the right positions and measurement conditions;
afterwards reading them out in the optimal way. All of this are defined clearly in every asset-specific package.

\paragraph{Installation and Commissioning}
Push the button of Industry 4.0. Initially links to the existing data sources are established and data gets ingested.
Where required a set of acquisition units and sensors is installed on the machine or structure.
After a final verification on the spot (SAT) the monitoring system is launched: the assets enter the \acl{IoT}.

\paragraph{Data Management}
Data is continuously streaming in from individual setups as well as historian sources.
Structuring, verifying and cleaning the data sets is an essential prerequisite to allow for a profound analysis afterwards: on your way to an automated, continuous and smart follow-up.

\paragraph{Analysis and Clear Reporting}
Advanced insights are unlocked using algorithms based on physics as well as big data approaches, as seen in Chapter~\ref{section:data_anal}
Clear dashboards, warnings and periodic reports inform the owner or line manager about the present state and upcoming issues. Surprises are avoided, standstills reduced.

\subsubsection{Flexibility}
Now that we are clear about the possible components of the \ac{SaaS}~\cite{Article:matthewlemerle_2012_its}, we can distinguish two possible different types of operational flow, as illustrated in figure~\ref{tik:standar_offer} and~\ref{tik:ligth_offer}.
\begin{figure}[ht]
	\begin{tikzcd}
		{Datasources} & Commissioning & {Management} & Analysis
		\arrow[tail, from=1-1, to=1-2]
		\arrow[tail, from=1-2, to=1-3]
		\arrow[tail, from=1-3, to=1-4]
	\end{tikzcd}
	\caption{Building blocks for light project}
	\label{tik:ligth_offer}
\end{figure}

The reasons for having two paths with separate initial phases are several. One important factor is the current global semi-conductor~\cite{Article:bbcnews_2021_chip} and chip shortage,
 which makes it difficult to procure \ac{IPCs} and sensors.
On the other hand the light package (Fig. ~\ref{tik:ligth_offer}) could represents a first step, to try the service with a limited effort.
Having said that, it has lower costs and timelines compared to the full package, but it complicates multi-source data management and the analysis that can be performed will be of lower granularity.
Table~\ref*{tab:phase_diff} provides a further comparison.
\begin{table}[h]
	\begin{tabularx}{\textwidth}{>{\parskip1ex}X@{\kern4\tabcolsep}>{\parskip1ex}X}
		\toprule
		\hfil\bfseries Full
		 &
		\hfil\bfseries Light
		\\\cmidrule(r{3\tabcolsep}){1-1}\cmidrule(l{-\tabcolsep}){2-2}
		%% FULL, separated by empty line or \par
		The engineering team has to design and devise which sensors to use, where to place them and why.\par
		Hardware initial cost is high, and it requires weeks or months before it is collected and delivered to the zensor offices.\par
		 &
		%% LIGHT, separated by empty line or \par
		Some sensors/devices are already inplace and can be integrated into the service\par
		Possibility to manage client's independently metric collected data, with an additional effort in the third step\par
		\\\cmidrule(r{3\tabcolsep}){1-1}\cmidrule(l{-\tabcolsep}){2-2}
		%% FULL, separated by empty line or \par
		Installation phase requires several steps, including testing hardware and connection both at the factory (FAT) and on-site (SAT)\par
		 &
		%% LIGHT, separated by empty line or \par
		Necessary and important information, such as network protocol and connection information, to set up links to existing data sources are collected.\par
		\\\bottomrule
	\end{tabularx}
	\caption{1° \texttt{Hardware} vs \texttt{Datasources} and 2° \texttt{Installation} vs \texttt{Commissioning} Phases differences}
	\label{tab:phase_diff}
\end{table}

That said, to give you some sense of what the service might do, here is a non-exhaustive list of the main advanced monitoring features and metrics available:
\begin{enumerate}
	\item[$\blacksquare$]\texttt{Availability}: have a continuous idea of availability, automatically as the platform combines different input streams and contextual information.
	\item[$\blacksquare$]\texttt{Performance}: based on the data collected and machine-learning based methods for determining the operational condition the performance is calculated.
	\item[$\blacksquare$]\texttt{Warnings}: whenever values start to deviate, or data streams stop, warnings are sent. This avoids ``black holes'' in the insights of the production line or assets.
	\item[$\blacksquare$]\texttt{Quality}: coupling to existing databases or using human input fields the product quality is linked to operational process parameters.
	\item[$\blacksquare$]\texttt{Factory information systems}: such systems are crucial for obtaining operational excellence. When well managed they maximize efficiency and effectiveness. Automated data collection and advanced analysis makes this possible.
\end{enumerate}

\subsubsection[short]{Maintenance metrics}
A typical analysis involves performing a variety of calculations to obtain useful metrics, from static to maintenance; but what are the maintenance metrics anyway? 
There are two categories of maintenance \ac{KPIs} which include the leading and lagging indicators. The leading indicators signal future events and the lagging indicators follow the past events.
We will give space to the latter category. For completeness here is how \acl{infra} platform could help you, asset owner, dealing with $MTTF$, $MTBF$, $OEE$.  
\begin{itemize}
	\item[$\star$] $MTTF$: The \textit{Mean Time Till Failure} is tracked continuously, for each asset covered the overall ``disturbance free'' operation is displayed.
	\item[$\star$] $MTBF$: As events and operating conditions are automatically detected the \textit{Mean Time Between Failures} is determined continuously, giving a good insight on where optimization is possible.
	\item[$\star$] $OEE$: Using all parameters cited above the \textit{Overall Equipment Effectiveness} is determined, a major parameter for optimizing asset management strategies and future investments, with a huge cost savings potential.
\end{itemize}