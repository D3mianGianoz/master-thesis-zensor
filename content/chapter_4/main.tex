\chapter{Use cases}\label{chapter:use_cases}

\section{Climbing the information Ladder and difference between data and information}
General note: I wouldn't go into too much detail. but I would tend to focus more on where I contributed to the success of the project, for better or worse.
\section{Monitor electricity consumption}
\textit{MOBI Hospital} -
\textit{Student Dorms} -
\textit{V U B Etterbeek Campus} -

One possible idea would be to compare the three different projects \textbf{20012/20013/21025} and highlight the commonalities, e.g. the purpose, the fact that they have the 'same' client [although I think they are different research groups], and the differences, such as the data collection and the target user of the dashboards or project age etc.

\paragraph{Vrije Universiteit Brussel}
The Vrije Universiteit Brussel is a Dutch and English-speaking research university located in Brussels, Belgium. Motto: \textit{"Conquering darkness by science"}

\section{Improve existing Industrial production}
\subsection{Analyse blade grinder vibration}
\textbf{21022:} Here I'd like to talk a little about the difficulties encountered in the vibration analysis, such as my lack of knowledge of the subject, doubts about the reference system (local vs global), the lack of real operational data and counter-intuitive results, and the need to strengthen the methodology and repeat the experiment \cite{Misc:stumabo_en_website}

% \input{./content/chapter_4/.tex}
\subsection{Increase efficiency of tomato company}
\textbf{21024}: Finally, I'd like to tell the tale of \cite{Misc:stoffels_en_website}:
\begin{itemize}
    \item how the demo of this project was launched with an intense daily sprint, followed by a duo collaboration with another inter (Kasia);
    \item how this project is a bit of a white fly compared to the zensor-standard project;
    \item  was also an opportunity for me to test my skills in the role of backend rather than analysis.
\end{itemize}
