\section{Monitor electricity consumption}
% Othe purpose, the fact that they have the 'same' client [although I think they are different research groups], and the differences,
Meaningful introduction:
such as the data collection and the target user of the dashboards or project age etc.

\subsection{Initial Hypothesis}\label{sub:vub_initial_hp}
\paragraph{Client} 
% with the motto: \textit{"Conquering darkness by science"} 
\ac{GEP} is a joint project of the \ac{VUB} (Free University of Brussels), dutch and English-speaking research university, and the \ac{UZB} (University Hospital of Brussels), 
which purpose is to develop and operate a research campus in the Research park of Zellik with a focus on the following three research domains:
\begin{itemize}
    \item Energy and mobility transition.
    \item Hospital of the future, part of \ac{BHC}.
    \item Smart regions.
\end{itemize}
With this research campus, Green Energy Park aims to bridge the gap between research, innovation, realization and exploitation, by acting as a large-scale living lab, expertise and training centre~\cite{Misc:vub_2020_green}.
\paragraph{Context Introduction}
\begin{figure}[ht]
    \fbox{\includegraphics[width=0.97\textwidth]{vub/context/campus_site_layout.pdf}}
    \caption{\acs{UZB} electricity distribution network layout}
    \label{fig:bhc_site_layout}
\end{figure}
As part of this project the \ac{BHC}, containing the academic hospital, is a well-advanced
energy island owning and running a state-of-the-art micro-grid that can work in island mode for five consecutive days. 
It includes a thermal and electricity grid, wastewater recovery, a high-
speed glass-fibre telecom network and a total of 33 \ac{HV} transformers divided over 18 \ac{HV} substations.
Energy production and storage includes solar PV, \ac{CHP}, three emergency diesel generators,
and a total capacity of 2,5 MWh in battery storage, mainly under the form of UPS.
The micro-grid serves the hospital complex, 250 student dwellings, the faculty of health
sciences, a primary school and a fitness centre. 
\begin{wrapfigure}{l}{.25\textwidth}
    \centering
    \includegraphics[width=.25\textwidth]{vub/context/trasformatori-uso-sopedaliero-300x253.jpg}
    % \caption{Example of an hospital transformer}
    % \label{fig:transformer_example}
\end{wrapfigure}
The micro-grid system is conceived to go in island mode with complete automatic transition in maximum 15 seconds 
in case of critical need and in three minute to comfort need. 
Cutting edge control technology and maximal reliability are the focus points of this demonstration site.
The hospital, our primary focus, has its own distribution network, as shown in Figure \ref{fig:bhc_site_layout}.
The topology of the network presents a closed-ring shape for increased reliability and is connected to the grid 
through two links to nodes C1A and C1B, located at the same place. Each ``node'' of the network is \ac{HV} cabin identified by a code \{C1, C2, \dots, C12\}, with a main transformer.
Furthermore it can have connection with one or more ``sub'' transformers, like the one showed in before, 
who in turn are connected to ``consumers'' or power sources. These can vary from an individual room to a whole medical department.
% or substations

\subsection{Goal(s), purpose \& critical factors}
\paragraph{Long term}
\begin{itemize}
    \item[$\circledcirc$] Grow Zensor into the main data hub for the Green Energy Park.
    \item[$\circledcirc$] Minimizing the energy losses and overall consumption, leading to a more profitable operation.
    \item[$\circledcirc$] Identify where the exact sources of this cost are and where the best opportunities for improvement lie.
\end{itemize}
\paragraph{Short term}
\begin{itemize}
    \item[$\circledcirc$] Having a view on the data, centralized and well accessible for multiple people in a structured way.
    \item[$\circledcirc$] Monitoring and tracking energy consumption in a production site resulting in more than mere energy bill reductions.
    \item[$\circledcirc$] Improve sustainability reducing energy need and peek request. 
\end{itemize}

\subsection{Project description by phases}
\begin{figure}[ht]
    \includegraphics[width=\textwidth]{vub/flowcharts/4_phases.png}
    \caption{\ac{VUB}'s (light) project core stages}
    \label{fig:vub_stages}
\end{figure}

% In this specific project the energy (related parameters) metered at the VUB BHC (the academic hospital in Jette) will have to be stored in our database and visualized for the client. 
% All the 'items' at that point should also have a link to the manual, spec-sheet... of the component (link to Snipe-It or other database).

\paragraph{Datasources}
\begin{figure}[ht]
    \includegraphics[width=\textwidth]{vub/flowcharts/folder_tree.png}
    \caption{\ac{VUB}'s remote server folder tree structure}
    \label{fig:vub_folder_tree}
\end{figure}
MOBI, \ac{VUB} research group, has collected through the years several MB of 15-minute data energy related out of the distribution network.
This limited dataset is currently stored on a remote server in a simple, yet effective way. 
a folder tree, that tries to reflect reality, is our main metadata's source. 
Looking at Figure \ref{fig:vub_folder_tree} from right to left, in a bottom up manner, should help us clarify the situation. %magari aggiungere colori ai nomi dei diversi livelli come in figura
Starting from the leaves, we found the ``\ac{csv} Files''. Each file share the same common structure with two columns, \textit{time} and \textit{value}, and multiple rows chronologically sorted.
It can either contain information about a secondary transformer or a consumer/power-source as mentioned before in Subsection \ref{sub:vub_initial_hp}. 
To make such distinction we have to rely on his filename, indicating the hardware source and/or the physical element measured.
For example we could find the \texttt{Transfo-I3.csv}, referring to transformer's 3° phase current or \texttt{Bord LG 03 Radiologie.csv} radiology department energy consumption.
How are the different physical quantities organized? To answer this question we go up one level in the tree, moving to the right, stepping up to ``Measurements''. 
Here we have five separated directory. These represent different electricity measurement $\{current, voltage, power\-factor, energy\}$, taken on the secondary \ac{LV} terminals of the transformers. 
About the energy (kWh): we have to make an important distinction between individual or cumulative. It can either represent the consumption of one individual consumer or the whole sub transformer 
(ideally it should be the sum of all consumer connected to it).
Instead, for the others (current, voltage e power factor data later added to the dataset), is only available at the sub transformer level, not consumer. Let's now change the way we traverse the tree, from bottom-up to top down.
From the root (``Fixed folder structure'') we go down to the ``Individual Nodes'' layer, which is pretty self-explanatory. We find in fact a directory for each node of our distribution network. 
This folder contains one or more sub-folder, one for each sub-transformer connected to the same node. Going down even further to the ``Sub transformer'' level, same logic applies. For each sub transformer we have it's one metrics folders.
So here it is the connection point.
To clarify, let's take the previous example and extend it further: the \texttt{Bord LG 03 Radiologie.csv} will have the following path \textit{root/NodeC03/Transfo0302/ConsumerEnergy/Bord\dots}.
%So each set of voltage-files is grouped in to a folder

\paragraph{Commissioning \& Data Management}
% Spostare i dati da un server all'altro e digerirli
\begin{figure}[ht]
    \includegraphics[width=\textwidth]{vub/flowcharts/data-management.png}
    \caption{\ac{VUB}'s data ingestion flowchart}
    \label{fig:vub_ingestion}
\end{figure}
Once the university granted the credential to remotely access this dataset, we started managing it, periodically accessing it using \ac{SFTP} over port 9921. % that is periodically
The \textbf{first step} of the ingestion is, as illustrated in the figure \ref{fig:vub_ingestion}, copying data from \ac{VUB}'s server to Zensor's one. 
This happens periodically as cron job, see Section \ref{subsection:script_structure}, also thanks to Python library \texttt{pysftp}. %enumerate?
We recursively traverses the sftp folder and if we comes across a .csv file we copies it into the same folder structure as on the server.

Subsequently, for \textbf{step two}, we close the connection and work with our freshly copied data. We traverse, once again, the folder tree, keeping track of
folder names that are gonna be important as metadata. Once we are at leaf level, see once again Figure \ref{fig:vub_folder_tree}, we use 
Pandas (\ref{section:pandas}) for file reading, inferring the datetime strings format. This switch to a faster parsing method can increase the speed by 5-10x. %parsing
Then we can perform some data cleaning, for eliminating outliers, duplicates and NaNs.
Immediately afterwards we tag the respective \textbf{DataFrame} with the necessary information to easily identify it, like description, unit and the metadata previously collected.

Since the numerous files are of significant size, this series of operations can take a long time. Therefore, it was decided to parallelize the whole operation using a pool of processes over threads.
To substantiate this point, here is a small digression: 
in Python, if code is CPU-bound, multithreading won't help, because only one thread can hold the Global Interpreter Lock, and therefore run Python code, at a time. 
So, in this specific scenario we need to use processes, not threads. Indeed in multiprocessing, any newly created process will run independently with its own memory space.

Finally, the \textbf{step three} involves writing each DataFrame on the InfluxDB (\ref*{section:influxdb}), our choice for \acl{tsdb}.
This operation, which is already quite optimized, can be easily executed in parallel given the limited concurrency.
As a result we will have a ``measurement'' for each node, which will contain several series, each one easily distinguishable from the others.

% In a second instance they want the entry point to be a map of the campus with the main transformers indicated and with a clickable link on each item such that the data can be consulted. 
\paragraph{Analysis \& Visualization}
\begin{figure}[ht]
    \includegraphics[clip, trim=0 0 0.1cm 3.5cm, width=\textwidth]{vub/grafana/node_c03_transfo302_raw.pdf}
    \caption{Transformer302 raw data dashboard}
    \label{fig:vub_raw_rad}
\end{figure}
To ensure that ingestion has been carried out correctly some basic visualizations is required.
We designed the dashboards to share the same structure, same philosophy as the distribution network.
So each node (\ac{HV} cabin) will have it's own page, with a variable of number of panels, depending on how many sub transformer are present per node.
Taking the NodeC03's dashboard, illustrated at Figure \ref{fig:vub_raw_rad}, as an example, with a bit of background info: 
this station has 3 sub-transformers and gives power to the children's hospital, as we can spot references to medical wards; 
we will focus on the second converter and its metrics as we can recognise some key elements previously discussed.
Starting from the bottom, we find three panels covering voltage, current and power factor of Transformer 302, with a minimum 6 month temporal frame.
As mentioned at the beginning of this segment, these measurements were added at a later stage and are in a smaller volume than the electric energy. 
At this stage of the project, there is not a high interest in performing any analyses on these metrics, besides trend monitoring. 
Directly above, a larger panel is showing the energy consumption of all consumers connected to the secondary transformer.
As we discussed in section \ref{section:grafana}, \textit{Grafana} allows us to adjust the time window to our liking.
In the time interval chosen, from July 2018 to March 2022, the signal appears to be a monotonously increasing function of time $f(t)$. 
In fact, the measured data, at the \ac{LV} stations, are cumulative: easy to collect and store, but less interesting from an analytical point of view. 
Following this argument, we are not surprised that, taking a random point (20 April 2021), the values recorded at that instant are of a higher order of magnitude [MWh] than expected [KWh].
% We will resume this issue at later stages.
\begin{figure}[ht]
    \includegraphics[width=\textwidth]{vub/flowcharts/analystic_flowchart.png}
    \caption{\ac{VUB}'s analytics chart}
    \label{fig:vub_anal_chart}
\end{figure}

It is important to remember that one of the objectives of this project is to provide information on the energy usage of the various nodes. 
The next step is then to calculate some consumption statistics. We now shall examine the workflow of operations necessary for achieving this goal.
Looking at the diagram in Figure \ref{fig:vub_anal_chart}, we can observe that the first necessary step is to choose the time period $T$ that most interests us.
Ideally, for the end user i.e.\ the vub researcher, all options will be available.
As for the subsequent steps, they are quite simple at an abstract level.
\begin{enumerate}
    \item First, following the previous discussion, we calculate how much the series varies. 
    That is, taking two successful points of the signal $x_t$ and $x_{t+1}$ we calculate $\Delta = x_{t+1} - x_t$.
    \item Second, We use pandas split-apply-combine approach, as discussed at \ref{fig:pandas_groupby}, we group on our tags columns and then we use the library downsampling 
    functionality for performing a frequency conversion, from 15-minutes data to the selected $T$ period.
    \item Third, we compute the group sum, as the apply step and we get our data combined as requested.
    \item Four, we decorate the resulting DataFrame with extra information, useful for visualization purpose, and the resulting frequency. 
\end{enumerate}

\begin{figure}[htp]
    \begin{subfigure}{\textwidth}
        \includegraphics[clip, trim=0 7.5cm 0 0, width=\textwidth]{vub/grafana/node_c03_consumption_stats.pdf}
        \caption{NodeC03 consumption statistic}
        \label{fig:vub_stats_v1}
    \end{subfigure}
\end{figure}
So all that remains is to visualise this statistic in an intelligent way, so that it can provide some insights for researchers.
We shall now take a look at two dashboard, illustrated at (\ref{fig:vub_2_dashboard}), with several panels querying the same consumption statistic.
As mentioned in the caption, we are displaying data concerning the preceding case, i.e. the radiology department of the children's hospital.
In detail, starting from (\subref{fig:vub_stats_v1}), on the top left we find the distribution of daily consumption, which is mainly between 20 and 80 kWh. 
On the right we can see the proportional average consumption per day of the week, where the maximum is Tuesday and the minimum Sunday. 
Finally, we have a status map, which gives us, at one glance, a comprehensive view of long-term average daily consumption. 
For instance, we were able to notice that on 28th December 2022 there was a spike in consumption.

Turning now to Figure (\subref{fig:vub_stats_v2}), we can see that this also has as many as 3 panels. Starting from the top left-hand corner, we have a histogram indicating the average energy consumption per day of the week, averaged over the selected period. 
Note that the y-axis does not start from 0, but from 37.5 to highlight the present differences. On the right we find a pie chart, similar representation to that discussed for panel (\subref{fig:vub_stats_v1}).
Finally, at the bottom we see a daily average over a longer period, i.e. a month. With peaks in June and July and lows in October. 
\begin{figure}    
    \ContinuedFloat
    \begin{subfigure}{\textwidth}
        \includegraphics[clip, trim=0 5.5cm 0 0, width=\textwidth]{vub/grafana/weekly_and_monthly_node_c3_transfo302_radio.pdf}
        \caption{stats}
        \label{fig:vub_stats_v2}
    \end{subfigure}
    \caption{Two examples of visualization dashboard of \ac{UZB} radiology's energy consumption}
    \label{fig:vub_2_dashboard}
\end{figure}
As a minor distinction here we have transparent panels, as contrasting with the white background of the previously mentioned dashboard. 
With these considerations we can consider the analysis part complete, knowing that a few aspects have been left out, in order to keep the text as relevant and clean as possible.

\subsection{Conclusion}
