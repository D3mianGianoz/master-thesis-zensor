\selectlanguage{italian}
\begin{center}
    \textbf{Sommario}
\end{center}
In questo documento viene discusso il problema 

\cleardoublepage

\selectlanguage{english}
\begin{center}
    \textbf{Abstract}
\end{center}

In today standard practice, the maintenance of assets is determined by the knowledge of a number of unique experienced individuals coupled with technical data sheets.
Product quality only becomes an issue when customers start complaining. Repairs are made when it is already far too late. All of this puts tremendous pressure on the people responsible for the production lines. Rapid developments in sensor technology, artificial intelligence and IoT have made predictive maintenance cheaper and easier to implement. These technologies enable a better way to validate the actual health of crucial assets. This is accomplished by monitoring continuously a broad data set, and by adding a specific multi-aspect monitoring setup consisting of different sensor types that follow the behaviour of critical components and the general state-of-health of the asset.

Zensor, the company hosting the internship, focuses its core business on monitoring and analysing data in critical environments. It provides a business intelligence service, consisting of 4 distinct building blocks, and makes it available to many types of businesses in various industries. This includes functions such as: reporting, online analytical processing, analytics, dashboard development, data mining, and so on.
In this document, we will report an overview on how to realize an asset management system, with some concrete use case, focusing on two components of the product: Data management \& Analysis.

\cleardoublepage