Nowadays, it's hard to go anywhere now without hearing about AI and machine learning and data, particularly. It's everywhere.
Research has suggested that every two years, we generate more data than ever existed before.
So the amount of data is doubling every two years now, that is absolutely an astronomical amount, 
but the thing is that, of course, this data doesn't necessarily mean anything. 
The fact is: you can create tables of data, but unless you understand what's in them and what they mean, you haven't got any knowledge.
Here we can see an important distinction between having \textbf{data} and having \textbf{knowledge}.
As a species, we're producing a huge amount of data, even if a lot of it doesn't get used, 
and it sits there on a hard disk waiting for someone to look at it.

If we want to extract knowledge from data we are going to need some tools and processes to do this in a formal way and
that's where data science comes into play.
So, perhaps, if you do this for your job, then data analysis could be useful for you.
Maybe your company's generating data, and you want to analyze this data?
On other hand perhaps you (reader) are just a consumer and companies are using data on you. 
They're generating data on you, and they're profiting from data on you. 
These are sometimes life-changing decisions that are being made on your data and so it's empowering to know how this process works.
Let's do a simple example:
suppose you go online to book some flights for a holiday,
and then you decide that actually, two flights via an intermediate airport is cheaper than a single flight;
for taking that decision you're doing data analysis, taking lots of different data sources and working out the optimal route.
This could also happen automatically as well depending on the flight website that you're using.
Let's try to formalize this process, what is the meaning of the different topics listed so far? 
One problem is that multiple definitions existed with a slight difference and, on the other hand, a lot of these terms are used completely interchangeably; 
AI is a classic example. You can't buy a product without it having been having AI added to it, when most of the time, 
the manufacturer are referring to machine learning.

The idea of AI is that we're training a machine to perform a task without explicitly programming it to do so. 
A good example of AI that isn't machine learning would be a mouse in a maze where all you're doing is telling 
it to turn left or right at random, not learning anything about the environment itself.
It doesn't understand what the maze is, but it will eventually get to the end right: that's a kind of rudimentary artificial intelligence
that doesn't involve learning anything
Another possible approach, Machine learning, is about \textbf{not} giving the mouse different operational conditions, but more about "feeding" it 
with examples and hoping it will learn to perform most tasks itself.
Hence, here is why machine learning is a subset of AI, and it shouldn't be used interchangeably to avoid user confusion.
If we use the latter what we'll end doing is training our model based on existing samples of data to either tease out information or make predictions on this data.
One of the main operational problems, that I could try in this experience, is that not all data is made nice equal; 
some of it's noisy and messy, maybe we don't know what it is and don't know whether we can apply a certain technique/idea to it.
And so from this come the necessity to clean this data up. This will involve taking this data, understanding what it is and extracting some knowledge 
so that we can then apply this AI or machine learning techniques to it.

Let's give to important informal definitions; Data science: take data and prepare it in a way that then it can be later used understood.
Data analysis: the idea of using statistical measures to try and work out what's going on
Perhaps, sometimes, just using statistics to analyze the data isn't enough; you can't learn everything about it, you can mathematically grasp how it works, but you might not understand what it all means.
So this is where visualizing the data can be really helpful, that's going to be charting it, plotting it, 
trying to work out trends and links between different variables. 
Analysis and visualization, jumping back and forth in a cycle, you could do both of these things numerous times, trying to work a way out

Another important aspect that we should talk about is Data pre-processing.
Often you'll be finding your recording much more data than you need. This is certainly true of an online shop, 
suppose I'm a random customer. As such I'm going to be looking at a lot of products, that I don't end up buying, and I was never really going to buy. 
In this case, the shop-owners have got to sort of 'weed out' this information to work out what it is that they might better convince me to buy.
So this is going to require preprocessing data and removing nonsense and drilling right down to the useful stuff.
This is pre-processing and is going to be in the loop together with analysis and visualization, as we can repeat these operations, 
drill-down and whittle down our data into the most usable sort of the core of knowledge that we can get the most out of it.

Now it may be the case that just analyzing the data is enough, but sometimes you want to take things a little further,
we could use machine learning or modelling, which perform two fundamental jobs. 
One is to classify data like - is there another car on the road? Or, Does this patient have cancer? 
The other is to make predictions about future outcomes like - will the stock go up? or, Which blog do you want to read next? for predicting what's going to happen next, 

Finally, there is data mining and big data.
I'm not sure what data mining is because I don't think anyone knows what is it. It's a bit of a buzzword
Really what data mining is a combination of pre-processing your data and may be using clustering to extract some knowledge from it
So that's our sort of it's a word that's come to be used in place of those things
If someone says they're doing data mining, that's what they're doing. It's a night it's a cool sounding word, but you're not mining anything, right?
You're just doing what everyone else does on data. 
Let's assume we've collected a lot of examples of a specific topic, a huge number, and each of our examples is quite complicated, it has a lot of variables
and so the amount of data we've got is sort of unwieldy, right?
I would argue, perhaps, that big data is not data that you can run on your laptop like you might be using cloud computing infrastructure or certainly parallel processing in some way to pre-process and analyze this data
So this is exactly where the line is, how Big Data is.