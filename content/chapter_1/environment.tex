not all data are the same, treated differently depending on the context. 
Increase the awareness of the problem

\subsection{Predictive maintenance}
~\cite{Misc:zensor_blog}
\subsection{Prescriptive maintenance}
\subsection{Repair minimization}

and mention the company at the end.
\subsection{A possible solution}
Today, most often, technical data sheets coupled with the knowledge of a number of unique experienced individuals are used to determine when the asset requires maintenance. 
Product quality only becomes an issue when customers start complaining and repairs are done when it's already far too late.
All of these puts tremendous strain on the people responsible for the asset, while it could be avoided.
Unexpected shutdowns are costly and very demanding for the workforce involved; a possible solution would be making assets smart in order to increase the availability.
The only way to validate the actual health is by having a continuous look at a broad data set and adding a specific multi-aspect monitoring setup 
consisting of different sensor types that follow the behaviour of the asset's general state-of-health.