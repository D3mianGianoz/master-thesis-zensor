\chapter{Concluding remarks}
I have argue through this work that there is a viable solution to improve the quality of maintenance. % the future of maintenance  
The number of industries picking up predictive maintenance keeps increasing every year, which is something we should all be excited about. 
It boosts productivity and therefore drives economic growth, it improves the durability of assets and helps us move to a more sustainable future creating new opportunities way beyond the maintenance sphere.
In particular I saw in \acl{cbm} a strong connection with the philosophy of the three R's - reduce, reuse and recycle - pushing industrial companies towards a direction to reduce waste, reuse assets and conserve natural resources. 

\paragraph{Review of the core ideas}
In this report, I have tried to explore the main components behind this peculiar service, starting with the theoretical fundamentals in Chapter \ref{chapter:intro}. 
In fact, I devoted myself to a preliminary literature search, which together with my previous studies, allowed me to delve a little into the rich world of data analysis. 
It was more challenging to approach the world of maintenance, on which I had few confused notions, but, thanks to many interactions with people who have been through this journey before, 
I managed to get an outlook on this sector and I do hope this will stand out.

In general, my daily experience at Zensor (\ref{chapter:zensor}) has been positive. I was able to make many valuable contributions to the organisation, both on a hard (specific assignments) and soft (team dynamics) level, integrating quickly into the team. 
The gap between the academic environment I was used to before and the the working environment was quite big, but I was able to bridge it faster than I expected.

The chapter I wrote chronologically earlier is the third one (\ref{chapter:tools}), since I needed to familiarise myself with the tools already in use, facing different challenges and learning from my failures.
What I have included in this document is an outline of the most valuable findings I would have liked to have at the start of my training.

During the internship I had to deal with projects that had a lot of engineering content that I was unfamiliar with. 
After an initial hesitation I managed to get used to the environment quite well and this allowed me to start delivering relevant results relatively swiftly, 
and the two cases discussed in chapter \ref{chapter:use_cases} should be proof of this fact.
% Dealing with data whose meaning is not known is a problem for several reasons...

\paragraph{Final thoughs}
This thesis project aims to challenge conventional ideas about the relationship between computer science and physics via electronics and engineering; seeing them as linked and co-productive processes. 
This experience has helped in developing a critical mindset, investigating and questioning the why and the how. I hope that, if further refined and applied with the right context, this approach will allow me to build a very rich professional life.
One very important and valuable lesson I have picked up is the relevance of the focus ``simplicity''. It is crucial to offer technological solutions, which include complex analysis, but at the end the key is to try to make sure that the outcome is simple and clear for the end consumer; 
this means integrating all the necessary tools and aspects that are often so complex and technologically advanced, but having the interface that the customer sees in the simplest way. 
You are only going to create an (industrial) revolution if your user actually understands what you are doing.
